
\chapter*{Úvod}

V posledných rokoch je z hľadiska urbánneho plánovania badať tendencie
znižovania akejsi nadvlády áut v uliciach. Najmä v krajských mestách
je zvyšovaný dôraz na budovanie infraštruktúr pre chodcov a cyklistov. Výnimkou nie je ani Bratislava,
v ktorej je v posledných rokoch možné pozorovať nárast práve prepravy po dvoch kolesách.
\newline
\par
Magistrát mesta aj z dôvodu ďalšej analýzy inštaloval od roku $2014$ na niektorých úsekoch cyklotrás 
sčítače, ktoré s hodinovou frekvenciou akumulujú absolútne počty cyklistov, ktorí v danom časovom intervale
prešli ktorýmkoľvek z dvoch smerov. Toto veľké množstvo dát vytvára priestor na hlbšie analýzy, ktoré bližšie
dokážu opísať správanie obyvateľov Bratislavy.
\newline
\par
V tomto reporte je cieľom poukázať na zaujímavé interpretačné dôsledky, vychádzajúce z dát. Dôraz je kladený na
interpretáciu dát s rôznou frekvenciou a spojitosti, resp. rozdielnosti medzi jednotlivými cyklotrasami. Okrem iného, hlbší
dôraz je kladený na spojitosť celkovej vyťaženosti cyklotrás s počasím. 
\newline
\par
Report vznikol ako jeden z podkladov riešenia semestrálneho projektu v rámci predmetu Princípy dátovej vedy
v zimnom semestri 2025–2026 na Fakulte matematiky, fyziky a informatiky Univerzity Komenského v Bratislave.


\begin{flushright}
\textit{v Bratislave, január 2026}
\par
autori
\end{flushright}

%% Toto sa bude dokončovať vzhľadom na obsah, ktorý bude

\tableofcontents

\clearpage
