
\chapter*{Úvod}
\addcontentsline{toc}{chapter}{Úvod}

V posledných rokoch je z hľadiska urbánneho plánovania možné badať tendencie
znižovania akejsi nadvlády áut v uliciach. Najmä v krajských mestách
je zvyšovaný dôraz na budovanie infraštruktúr pre chodcov a cyklistov. Výnimkou nie je ani Bratislava,
v ktorej je v posledných rokoch možné pozorovať nárast práve prepravy po dvoch kolesách. Pre úplnosť – po dvoch kolesách bez motora :)
\newline
\par
Magistrát mesta aj z dôvodu ďalšej analýzy inštaloval od roku $2014$ na niektorých úsekoch cyklotrás 
sčítače, ktoré s hodinovou frekvenciou akumulujú absolútne počty cyklistov, ktorí v danom časovom intervale
prešli ktorýmkoľvek z dvoch smerov. Toto veľké množstvo dát vytvára priestor na hlbšie analýzy, ktoré bližšie
dokážu opísať správanie cyklistov v Bratislave.
\newline
\par
V tomto reporte je cieľom poukázať na zaujímavé interpretačné dôsledky vychádzajúce z dát. Dôraz je kladený na
interpretáciu dát s rôznou frekvenciou a na spojitosti, resp. rozdielnosti medzi jednotlivými cyklotrasami. Postupne, v jednotlivých kapitolách
reportu, je kladený dôraz na analýzu vplyvu vybraných kvantitatívnych vlastností počasia na vyťaženosť bratislavských cyklotrás. Bližšie sa preukazuje 
vyšší vplyv zmien počasia na špecifickú skupinu rekreačných, resp. provinčných, cyklotrás.
\newline
\par
Špecifický dôraz je kladený na cyklotrasu na Vajanského nábreží, ktorej osadenie, ktorému predchádzalo zúženie doterajšej cestnej komunikácie o jeden pruh,
vyvoláva aj dodnes výraznú spoločenskú diskusiu. Na základe veľkého množstva dát bolo možné preukázať viaceré hypotézy, ktoré do značnej miery odpovedajú na otázky
opodstatnenia jej osadenia.
\newline
\par
V neposlednom rade je dôraz kladený na analýzu krátkodobých, resp. dlhodobých, trendov vyťaženosti cyklotrás. Je možné pozorovať viacero menej aj viac zjavných 
špecifík priebehu vyťaženosti počas dní, týždňov až jednotlivých mesiacov roka.
\newline
\par
Report vznikol ako jeden z podkladov riešenia semestrálneho projektu v rámci predmetu Princípy dátovej vedy
v zimnom semestri 2025–2026 na Fakulte matematiky, fyziky a informatiky Univerzity Komenského v Bratislave.


\begin{flushright}
\textit{v Bratislave, január 2026}
\par
autori
\end{flushright}

\tableofcontents

\clearpage
