\chapter[Analýza vplyvu novej cyklotrasy na Vajanského nábreží \\ \small Tuan Dávid Nguyen Van]{Analýza vplyvu novej cyklotrasy na Vajanského nábreží}

V tejto kapitole budeme analyzovať vplyv cyklotrasy na Vajanského nábreží v Bratislave, ktorá bola otvorená 1. septembra 2023.
Táto trasa spája mestské centrum s rekreačnými oblasťami a zároveň poskytuje alternatívnu cesty pre dochádzajúcich cyklistov.

Cieľom tejto kapitoly je kvantifikovať kauzálny vplyv otvorenia cyklotrasy na Vajanského nábreží na intenzitu cyklistickej 
dopravy a identifikovať charakter jej využitia. Konkrétne sa zameriavame na odpovede na nasledujúce výskumné otázky: Viedlo 
otvorenie cyklotrasy k štatisticky signifikantnému nárastu počtu cyklistov? Aký je časový profil tohto vplyvu? Jedná sa o okamžitý 
skok alebo postupný nárast? Aké sú charakteristiky využitia novej trasy v porovnaní s typickými dochádzkovými a rekreačnými trasami v meste?

\section{Dáta}

Na cyklotrase na Vajanského nábreží sú nainštalované dva paralelné sčítače (Vajanského 1 a Vajanského 2). 
Keďže tieto sčítače monitorujú tú istú infraštruktúru, ich hodnoty boli v každom časovom okamihu sčítané a ďalej analyzované ako jedna trasa „Vajanského“.
Sčítače na Vajanského nábreží poskytujú dáta až od konca januára 2024. Do osadenia nových sčítačiek na Vajanského 
mesto vychádzalo z čísel cyklosčítačiek pri River Parku. Pre zachytenie vývoja cyklistickej dopravy pred 
týmto obdobím bola využitá lokalita River Park, ktorá sa nachádza v bezprostrednej blízkosti analyzovanej trasy a vykazuje 
dlhodobú dostupnosť dát už od roku 2022. 

Na porovnanie správania cyklistov na rôznych typoch trás boli do analýzy zahrnuté aj ďalšie dve lokality:
\textbf{Páričkova}, reprezentujúca predovšetkým dochádzkovú cyklistickú dopravu a \textbf{Viedenská}, reprezentujúca rekreačnú cyklistickú trasu.
Meteorologické dáta boli získané zo služby Meteostat pre lokalitu Bratislava. Použité premenné zahŕňajú priemernú dennú teplotu, úhrn zrážok a priemernú 
rýchlosť vetra.

\begin{figure}[H]
    \centering
    \includegraphics[width=0.7\linewidth]{images/trasy.png}
    \caption{Cyklotrasy}
    \label{obr2.1}
\end{figure}

\subsection{Čistenie a príprava dát}
Pôvodné dáta zo sčítačov sú dostupné v hodinových intervaloch a rozlišujú smer jazdy.
Pre každý záznam bol preto najskôr vypočítaný celkový počet cyklistov ako súčet prejazdov v oboch smeroch.
Následne boli dáta agregované na dennú úroveň, čo umožňuje eliminovať krátkodobé fluktuácie v rámci dňa
a zároveň zabezpečuje kompatibilitu s dennými meteorologickými údajmi.

V prípade Vajanského nábrežia boli dáta z oboch paralelných sčítačov transformované do spoločného časového formátu
a následne sčítané, čím vznikol jednotný denný časový rad reprezentujúci intenzitu cyklistickej dopravy na tejto trase.
Keďže sčítače na Vajanského nábreží sú dostupné až od konca januára 2024, pre obdobie pred ich inštaláciou
bol ako referenčná lokalita použitý sčítač River Park.
Denné dáta z River Parku boli použité výlučne do momentu, keď sa začínajú objavovať dáta zo sčítačov na Vajanského nábreží,
aby sa predišlo prekrývaniu rôznych zdrojov merania.

Takto skonštruovaný časový rad umožňuje analyzovať vývoj cyklistickej dopravy na Vajanského nábreží
v dlhšom časovom horizonte, pričom zmena meracej lokality nevedie k diskontinuite v dátach.
Predpokladom tohto prístupu je, že River Park a Vajanského nábrežie zachytávajú podobné dopravné vzorce,
keďže ide o priestorovo blízke úseky tej istej dopravnej osi.

Dátum otvorenia cyklotrasy na Vajanského nábreží bol stanovený na 1. september 2023.
Na identifikáciu potenciálneho vplyvu tejto udalosti bola vytvorená binárna premenná,
ktorá nadobúda hodnotu 1 v období po otvorení cyklotrasy a 0 v období pred ním.
Okrem toho bola skonštruovaná trendová premenná vyjadrujúca počet dní od otvorenia cyklotrasy,
ktorá umožňuje zachytiť postupnú adaptáciu cyklistov na novú infraštruktúru.

Meteorologické údaje boli časovo zosúladené s cyklistickými dátami na dennej báze
a spojené pomocou dátumu.
Do výsledného datasetu boli zahrnuté priemerná denná teplota, úhrn zrážok a priemerná rýchlosť vetra,
ktoré predstavujú kľúčové faktory ovplyvňujúce rozhodnutie o využití bicykla.

Na kontrolu rozdielov medzi pracovnými dňami a víkendmi bola vytvorená binárna premenná,
ktorá nadobúda hodnotu 1 pre soboty a nedele a 0 pre pracovné dni.
Táto premenná umožňuje oddeliť rekreačné cyklistické správanie od každodennej dochádzky.

Výsledkom procesu čistenia a prípravy dát je konzistentný denný časový rad,
ktorý kombinuje dáta z viacerých sčítačov, explicitne zohľadňuje otvorenie cyklotrasy
a zároveň kontroluje vplyv základných meteorologických a kalendárnych faktorov.

\section{Regresný model}

Analýza bola realizovaná pomocou lineárnych regresných modelov odhadovaných metódou najmenších štvorcov  
na denných agregovaných dátach. Keďže analyzované dáta majú charakter denného časového radu, nie je splnený 
predpoklad nezávislosti a konštantného rozptylu rezíduí vyžadovaný klasickým OLS modelom. Pri dátach o 
cyklistickej doprave možno očakávať autokoreláciu medzi po sebe nasledujúcimi dňami aj rozptyl rezíduí 
spôsobenú sezónnosťou. Z tohto dôvodu boli použité HAC smerodajné chyby podľa Newey--West, ktoré poskytujú 
konzistentné odhady aj v prítomnosti autokorelácie a rozptyl neznámeho tvaru. Počet oneskorení 
bol nastavený na 7 dní, aby zachytil týždenný cyklus v správaní cyklistov.

\subsection{Model pre trasu Vajanského nábrežie}
Pre trasu Vajanského nábrežie bol použitý model s prerušovaným trendom, ktorého cieľom je zachytiť zmenu 
vývoja cyklistickej dopravy po otvorení novej cyklistickej infraštruktúry v septembri 2023.
Model zahŕňa:

\begin{itemize}
    \item lineárny časový trend (\texttt{days\_since\_start}),
    \item binárnu premennú indikujúcu obdobie po otvorení cyklotrasy (\texttt{post\_opening}),
    \item premennú umožňujúcu zmenu sklonu trendu po otvorení (\texttt{days\_after\_opening}),
    \item meteorologické kontrolné premenné (teplota, zrážky, rýchlosť vetra),
    \item indikátor víkendových dní,
    \item mesačné indikátory na kontrolu sezónnosti.
\end{itemize}

\subsection{Porovnávacie modely pre trasy Páričkova a Viedenská}
Pre trasy Páričkova (dochádzková trasa) a Viedenská (rekreačná trasa) bol použitý jednoduchší regresný 
model bez časového trendu a bez intervencie, keďže na týchto trasách nedošlo k porovnateľnej infraštruktúrnej 
zmene v analyzovanom období.

Vysvetľujúce premenné v týchto modeloch zahŕňajú:

\begin{itemize}
    \item meteorologické kontrolné premenné (teplota, zrážky, rýchlosť vetra),
    \item indikátor víkendových dní,
    \item mesačné indikátory na kontrolu sezónnosti.
\end{itemize}

Tieto modely slúžia primárne na porovnanie citlivosti rôznych typov cyklistických trás na počasie a sezónnosť, 
nie na identifikáciu kauzálneho efektu infraštruktúrnej zmeny.

\section{Výsledky}
Kľúčovým zistením analýzy je štatisticky signifikantný pozitívny efekt premennej \texttt{days\_after\_opening}, 
zachytávajúcej zmenu trendu po otvorení cyklotrasy. Odhadnutý koeficient tejto premennej dosahuje hodnotu $0.912$ 
s p-hodnotou $p = 0.039$, čo indikuje, že po otvorení novej infraštruktúry v septembri 2023 sa denný počet cyklistov 
zvyšuje v priemere o približne jedného cyklistu za každý deň. Tento postupný nárast naznačuje, že cyklisti si novo 
vybudovanú trasu osvojujú postupne v čase. Kumulatívny efekt tohto postupného nárastu je podstatný, po sto dňoch 
od otvorenia predstavuje tento trend približne 91 dodatočných cyklistov denne v porovnaní s obdobím bezprostredne po otvorení.

Binárna premenná \texttt{post\_opening}, indikujúca obdobie po otvorení cyklotrasy, vykazuje pozitívny koeficient 
$181.9$, avšak tento efekt je na hranici štatistickej signifikancie s p-hodnotou $p = 0.079$. Tento výsledok 
možno interpretovať ako okamžitý nárast využitia bezprostredne po otvorení cyklotrasy, hoci je potrebné pristupovať 
k tomuto zisteniu s opatrnosťou vzhľadom na marginálnu štatistickú signifikanciu. 

Kombinácia oboch efektov, okamžitého nárastu a postupného trendu, naznačuje, že otvorenie cyklotrasy malo dvojfázový 
vplyv na cyklistickú dopravu, kde sa počiatočný skok v návštevnosti kombinoval s následným postupným nárastom využitia.

Priemerná denná teplota má na Vajanského nábreží výrazný pozitívny efekt. Zvýšenie teploty o 1~°C je spojené s nárastom 
približne o 43 cyklistických prejazdov denne, pričom tento efekt je vysoko štatisticky významný. Naopak, zrážky a vietor pôsobia negatívne. 
Každý milimeter zrážok znižuje denný počet cyklistov približne o 35, zatiaľ čo zvýšenie rýchlosti vetra o 1~m/s vedie k poklesu o približne 
13 prejazdov denne.

Mesačné indikátory zachytávajú výraznú sezónnosť v cyklistickej doprave. Najvyššia aktivita sa pozoruje v jarných a letných 
mesiacoch, pričom máj a jún vykazujú najvyššie koeficienty okolo 400 cyklistov v porovnaní s januárom ako referenčným 
mesiacom. Jesenné a zimné mesiace vykazujú podstatne nižšiu aktivitu, pričom december dokonca vykazuje negatívny koeficient 
v porovnaní s januárom, hoci tento efekt nie je štatisticky signifikantný.

Zaujímavým špecifikom Vajanského nábrežia je jeho hybridný charakter. Analýza neukázala štatisticky významný rozdiel medzi 
pracovnými dňami a víkendmi, čo znamená, že trasa je rovnomerne využívaná počas celého týždňa a slúži teda rovnako intenzívne 
na dochádzanie do práce, ako aj na voľnočasové a rekreačné jazdy.

\subsection{Porovnanie s dochádzkovými a rekreačnými trasami}

Pre kontextualizáciu výsledkov z Vajanského nábrežia boli analyzované aj dve ďalšie trasy s odlišným charakterom využitia: 
Páričkova ako typická dochádzková trasa a Viedenská ako rekreačná trasa. Tieto porovnávacie analýzy umožňujú lepšie pochopiť 
špecifiká novej cyklotrasy na Vajanského nábreží.

Kľúčovým rozlišovacím znakom pri modeli pre trasu Páričkova je výrazne negatívny koeficient víkendového indikátora na úrovni 
$-203.66$, ktorý je vysoko štatisticky signifikantný. Tento pokles o viac než 200 cyklistov počas víkendov v porovnaní s pracovnými 
dňami predstavuje takmer 60-percentný pokles aktivity a jasne potvrdzuje primárne dochádzkovú funkciu tejto infraštruktúry. 
V kontraste s týmto výsledkom Vajanského nábrežia nevykazuje štatisticky signifikantný rozdiel medzi pracovnými dňami a víkendmi, 
čo naznačuje jeho zmiešanú funkciu.

Víkendový indikátor modelu pre Viedenskú trasu má pozitívny koeficient 198.19, ktorý je vysoko štatisticky signifikantný s prakticky nulovou p-hodnotou. 
Tento nárast takmer 200 cyklistov počas víkendov v porovnaní s pracovnými dňami predstavuje približne 84-percentný nárast oproti 
baseline úrovni a jednoznačne potvrdzuje rekreačný charakter tejto trasy. Tento výsledok stojí v dokonalom kontraste s Páričkovou, 
kde víkendy znamenajú pokles aktivity, zatiaľ čo na Viedenskej predstavujú vrchol využitia.

Na Páričkovej koeficient teploty dosahuje hodnotu $22.87$, čo pri baseline úrovni približne 341 cyklistov predstavuje relatívny 
nárast približne 6.7~\% pri zvýšení teploty o jeden stupeň Celzia. Na Viedenskej je koeficient teploty $33.05$, čo pri baseline 
úrovni približne 236 cyklistov predstavuje relatívny nárast približne 14~\%. Podobný vzorec 
sa pozoruje aj pri zrážkach, kde Viedenská vykazuje pokles približne 11~\% na milimeter zrážok, v porovnaní s 5.6~\% na Páričkovej. 
Rýchlosť vetra má na Viedenskej taktiež výraznejší relatívny vplyv než na dochádzkovej trase.

Vajanského nábrežie vykazuje meteorologickú citlivosť, ktorá sa nachádza medzi týmito dvoma extrémami. Koeficient teploty $42.96$ 
pri baseline úrovni približne 617 cyklistov predstavuje relatívny nárast približne 7~\%, čo je porovnateľné s dochádzkovými trasami. 
Táto podobnosť v relatívnych efektoch naznačuje, že hoci Vajanského nábrežie má vyššiu absolútnu baseline úroveň využitia, citlivosť 
používateľov na poveternostné podmienky je bližšie dochádzkovému než rekreačnému správaniu. V kombinácii s absenciou štatisticky 
signifikantného víkendového efektu to naznačuje, že trasa slúži predovšetkým pravidelným používateľom, ktorí ju využívajú konzistentne 
počas celého týždňa.

Sezónne vzorce zachytené mesačnými indikátormi vykazujú na všetkých troch trasách podobný kvalitatívny priebeh s maximom v jarných a 
letných mesiacoch, hoci s odlišnou intenzitou. Viedenská vykazuje najvýraznejšiu sezónnosť, zatiaľ čo Páričkova a Vajanského nábrežie vykazujú mierne 
nižšiu sezónnu amplitúdu, čo opäť potvrdzuje, že pravidelní dochádzkoví cyklisti využívajú infraštruktúru počas celého roka, hoci s 
nižšou intenzitou v zimných mesiacoch.


Porovnanie týchto troch lokalít poskytuje dôležitý kontext pre interpretáciu výsledkov z Vajanského nábrežia. Absencia 
signifikantného víkendového efektu a podobná relatívna citlivosť na meteorologické podmienky ako na dochádzkovej trase 
Páričkova naznačujú, že Vajanského nábrežie slúži ako hybridná trasa kombinujúca dochádzkovú a rekreačnú funkciu. Táto 
dualita využitia môže čiastočne vysvetľovať silný pozitívny trend po otvorení cyklotrasy. Nová infraštruktúra priláka 
nielen dochádzajúcich cyklistov hľadajúcich bezpečnejšiu a príjemnejšiu cestu do práce, ale aj rekreačných cyklistov 
využívajúcich atraktívne prostredie pozdĺž Dunaja. Schopnosť novej infraštruktúry osloviť obe skupiny používateľov môže 
byť kľúčovým faktorom jej úspechu a mohla by slúžiť ako model pre budúce projekty cyklistickej infraštruktúry v meste.
