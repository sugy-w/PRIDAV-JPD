\chapter[Klasifikácia cyklotrás na základe vyťaženia {\small Jakub Novotný}]{Klasifikácia cyklotrás na základe vyťaženia}
Cieľom tejto kapitoly je zistiť, či môžeme určiť konkrétny názov cyklotrasy len na základe toho, ako sa využíva. 
Každé miesto v Bratislave má svoje vlastnosti – niektoré sú hlavne na dochádzanie do práce 
(sú tu veľké návaly ráno a poobede počas pracovných dní), iné sú viac na voľný čas (viac ľudí cez víkendy a poobede). 
V tejto časti použijeme viacero metód strojového učenia, konkrétne logistickú regresiu, náhodné lesy a neurónovú sieť.
Úspešná klasifikácia by ukázala, že jednotlivé cyklotrasy majú jedinečné a predvídateľné `správanie'.


\section{Predspracovanie dát}
Na klasifikáciu sme surové dáta upravili na formát, ktorý môžu použiť všetky použité metódy, vrátane neurónovej sieti. 
Tento proces mal tieto kroky:
\begin{itemize}
    \item \textbf{Agregácia a čistenie}: Dáta sme prepočítali na hodinové intervaly a chýbajúce hodnoty nahradili nulou. 
    Odstránili sme všetky týždne, kde neboli kompletné záznamy, aby sme predišli skresleniu.
    \item \textbf{Vytváranie sekvencií}: Dáta sme rozdelili na týždenné úseky (168 hodín). Vstup pre jednu vzorku obsahuje
    kombináciu oboch smerov jazdy („Z“ a „Do“), čo dáva $168+168$ hodnôt pre každú klasifikovanú vzorku.,
    \item \textbf{Normalizácia}: Všetky vstupné hodnoty sme štandardizovali pre \textit{logistickú regresiu} a
    škálovali na interval $[0,1]$ pre \textit{neurónovú sieť}.
    \item \textbf{Encoding}: Názov cyklotrasy sme najprv očíslovali pomocou \textit{LabelEncoder}
    a potom previedli na pravdepodobnosti pomocou \textit{to\_categorical} z knižnice \textit{keras}.
\end{itemize}

\section{Použité modely}
V tejto časti sme testovali, ako dobre rôzne algoritmy dokážu klasifikovať cyklotrasy. Začali sme jednoduchými modelmi, ktoré spracovávali dáta po jednotlivých záznamoch, a prešli sme k zložitým neurónovým sieťam, ktoré spracovávajú časové rady ako celok.

Na rozdiel od predchádzajúcich kapitol sme tentokrát nezohľadnili počasie, aby sme to zjednodušili. Chceli sme zistiť, či môžeme trasu určiť len na základe vnútorných vzorcov vyťaženosti (hodina, deň, počet cyklistov).

Modely, najmä tie jednoduchšie, sa snažili dosiahnuť úspech tým, že predikovali najčastejšie sa vyskytujúce trasy,
čo skresľovalo presnosť. Model sa `naučil', že najlepšie je tipovať trasy s najväčším počtom záznamov, preto sme pristúpili k využívaniu \textbf{váhovania tried}.

\subsection{Logistická regresia (základný model)}
Ako základ sme použili logistickú regresiu. Tento model sme trénovali na pôvodnej štruktúre dát, kde každý riadok bol samostatný záznam.
Vstupné údaje: ['POCET\_Z', 'POCET\_DO', 'hodina', 'den\_v\_tyzdni', 'mesiac']. Časové údaje sme vzali z dátumu a času.

Model dosiahol nízku presnosť ($19\%$). To značí, že posudzovať jednotlivé hodiny zvlášť nestačí, 
pretože mnohé trasy majú v určitých hodinách podobný počet cyklistov a ťažko sa dá vyčítať nejaké trendy. 
Zároveň je na Obrázku \ref{obr:cm_lgr1} možné pozorovať, že model sa skôr naučil početnosti trás, 
pretože predikoval najčastejšie trasy v dátach, dokonca sa vôbec nesnažil predikovať tie menej vyskytujúce. 
To nám ukázalo problém nevývaženosti dát, v matici nie je takmer nič na diagonále - tzn. veľmi malú nepresnosť.

\begin{figure}[H]
    \centering
    \includegraphics[width=0.8\linewidth]{images/logistic\_regression\_cm1.png}
    \caption{Matica zámen pre logistickú regresiu bez hodinových okien.}
    \label{obr:cm_lgr1}
\end{figure}

\subsection{Random Forest Classifier}
Aby sme zachytili zložité vzťahy v dátach, použili sme aj algoritmus náhodného lesa. 
Použili sme rovnaké údaje ako pri logistickej regresii, aby sme mohli porovnať výsledky.
Síce sa presnosť modelu zvýšila na $29\%$, model stále nedokázal spoľahlivo rozlíšiť medzi trasami, čo ukázalo potrebu prejsť na analýzu časových radov.
Na Obrázku \ref{obr:cm_rfc1} je možné pozorovať jemnú diagonalizáciu matice zámen, ale stále vidno značné `svetlejšie' vertikálne čiary nad
\textit{Viedenskou / Hrádzou} (zberané už 11 rokov), zatiaľ čo nad \textit{Einsteinovou} (zberané približne rok) je `tmavo'.

\begin{figure}[H]
    \centering
    \includegraphics[width=0.8\linewidth]{images/random\_forest\_cm1.png}
    \caption{Matica zámen pre náhodné lesy bez hodinových okien.}
    \label{obr:cm_rfc1}
\end{figure}



\subsection{Neurónová sieť (LSTM)}
V treťom kroku sme prešli na hlboké učenie a využili sme predspracované týždenné okná. Model bol vytvorený pomocou Keras/TensorFlow a obsahoval tieto časti:
\begin{itemize}
    \item \textbf{Bidirectional LSTM:} Hlavná časť modelu - učí sa dlhodobé vzťahy v časových radoch a spracováva sekvenciu v oboch smeroch.
    \item \textbf{Dropout:} Technika, ktorá počas tréningu náhodne vypína niektoré neuróny, aby sme predišli preučeniu a donútili model učiť sa lepšie vzory.
\end{itemize}

\section{Výsledky klasifikácie a vyhodnotenie}
Po transformácii dát na týždenné časové okná a natrénovaní modelov sme pristúpili k vyhodnoteniam ich úspešnosti.
Zatiaľ čo celková presnosť nám poskytuje rýchly prehľad, analýza chybovosti pomocou matice zámen nám pomáha s identifikáciou vzorcov chybovosti. 
Na nasledujúcom obrázku \ref{obr:cm_lgr2} môžeme pozorovať, ako model klasifikoval jednotlivé testovacie vzorky oproti ich skutočným označeniam.

\begin{figure}[H]
    \centering
    \includegraphics[width=0.8\linewidth]{images/neural\_network\_cm.png}
    \caption{Matica zámen pre neurónovú sieť s hodinovými oknami.}
    \label{obr:cm_lgr2}
\end{figure}

Z tejto vizualizácie môžeme mať radosť, pretože `svetlé' štvorčeky sú prevažne na diagonále, čo zn. lepšia presnosť