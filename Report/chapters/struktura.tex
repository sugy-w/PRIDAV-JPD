\chapter{Štruktúra repozitára}

V prvom rade je vhodné ozrejmiť štruktúru repozitára,
v ktorom sú dodané všetky zdrojové kódy a súbory, ktoré
slúžili ako podklady k riešeiu otázok v tomto reporte.

Každá samostatná kapitola, ktorá rieši určitú otázku,
má svoj vlastný adresár v repozitári. Jednotlivé adresáre prináležia k
kapitolám nasledovne:
\begin{itemize}
    \item \texttt{/Weather Regression} -- Kapitola 2: Spojitosť vyťaženia cyklotrás a počasia
    \item \texttt{/TemporalRoutePatterns} -- Kapitola 3: Vyťaženosť cyklotrás v rámci dňa, týždňa, mesiaca, roku
    \item \texttt{/Routes Classifier} -- Kapitola 4: Klasifikácia cyklotrás na základe vyťaženia
    \item \texttt{/Vajanského nábrežie} -- Kapitola 5: Analýza vplyvu novej cyklotrasy na Vajanského nábreží
    
\end{itemize}

Okrem toho je súčasťou repozitára aj pomocný adresár \texttt{/Visuals of bicycle routes}, ktorý obsahuje vizualizácie všetkých analyzovaných cyklotrás adresár adresár a
zároveň aj zdrojové kódy, ktoré pomocou knižnice \texttt{osmnx} slúžia ako zdroj vizuálov.