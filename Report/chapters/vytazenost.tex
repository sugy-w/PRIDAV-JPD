\chapter[Vyťaženosť cyklostrás v rámci dňa, týždňa, mesiaca, roku \\ \small Lucia Ganajová]{Časová vyťaženosť cyklotrás}

V tejto kapitole sa zameriame na analýzu časovej vyťaženosti cyklotrás z rôznych časových perspektív – počas dňa, týždňa, mesiaca a roka. Cieľom je preskúmať, ako sa intenzita cyklistickej dopravy mení v priebehu týchto časových intervalov a či sú cyklotrasy využívané rovnomerne, alebo či medzi nimi existujú výrazné rozdiely.

\section{Dáta}
Pre analýzu časovej vyťaženosti sme pôvodné dáta rozšírili o viaceré časové atribúty, ktoré nám umožňujú podrobnejší pohľad na vyťaženosť cyklotrás.
Medzi pridelenými atribútmi sú:
\begin{itemize}
    \item \code{hour} – hodina dňa

    \item \code{dayofweek} – deň v týždni

    \item \code{dayofyear} – poradové číslo dňa v roku

    \item \code{month}, \code{quarter}, \code{year} – pre sledovanie dlhodobejších trendov

    \item \code{is\_weekend} – binárny indikátor víkendu

    \item \code{is\_winter}, \code{is\_summer}, \code{is\_spring}, \code{is\_fall} - indikátory jednotlivých ročných období
\end{itemize}

Hodina dňa je kľúčová pre analýzu vyťaženosti v rámci jedného dňa, zatiaľ čo atribút \code{is\_weekend} nám pomáha rozlíšiť vzory medzi pracovnými dňami a víkendmi, ktoré sú spravidla odlišné. Ďalšie časové ukazovatele nám umožňujú sledovať sezónne a ročné trendy.


\section{Vyťaženosť počas dňa}

Najmenšou sledovanou časovou mierkou je deň, preto začneme touto analýzou. Už na prvý pohľad je na Obrázoku \ref{weekday_vs_weekend_robust}) vidieť výrazný rozdiel medzi vyťaženosťou počas pracovných dní a víkendov. Cez víkendy sa na cyklotrasách bezpochybne nechádza viac cyklistov ako počas týždňa. Ďalším rozdielom je, že počas pracovných dní majú počty cyklistov dva výrazné vrcholy: ráno okolo šiestej hodiny a popoludní okolo šestnástej hodiny, čo naznačuje hlavne dochádzkovú dopravu do práce či školy. Naopak počas víkendov je počet cyklistov všeobecne vyšší a rozložený rovnomernejšie počas celého dňa, čo môže súvisieť s rekreačným využívaním cyklotrás.

\begin{figure}[H]
    \centering
    \includegraphics[width=0.7\linewidth]{images/weekend_vs_weekdays_robust.png}
    \caption{Porovnanie priemernej vyťaženosti cyklotrás v rámci dňa počas pracovných dní a víkendu.}
    \label{weekday_vs_weekend_robust}
\end{figure}

\subsection{Vyťaženosť jednotlivých trás}

Obrázok \ref{weekday_vs_weekend_routes} rozlišuje vyťaženosť podľa jednotlivých cyklotrás počas pracovných dní a víkendov. Väčšina trás vykazuje rovnaký vzor – počas pracovných dní sú využívané najmä na dochádzku do práce či školy. Samozrejme existujú aj výnnimky: Dunajská cyklotrasa alebo Cyklomost Slobody sa nevyznačujú dvoma výraznými vrcholmi počas pracovných dní, čo naznačuje skôr rekreačné využitie, nie dochádzkové. K týmto trasám patrí aj najvyťaženejšia zo všetkých - Dolnozemská cyklotrasa, ktorá je vytažená počas celého týždňa, najmä v popoludňajších hodinách a taktiež počas víkendov, no to počas celého dňa.

Zaujímavé je tiež pozorovanie, že počas pracovných dní je popoludňajší vrchol (okolo šestnástej hodiny) výraznejší než ranný. Hoci by sa mohlo zdať, že viac ľudí používa bicykel na cestu z práce, tento efekt pravdepodobne súvisí s vyšším počtom rekreačných cyklistov, ktorí sa po pracovnom dni venujú pohybu na bicykli.

Na základe dochádzkových vrcholov by sme jednotlivé trasy vedeli klasifikovať na primárne rekreačné a primárne dochádzkové.



\begin{figure}[H]
    \centering
    \includegraphics[width=0.9\linewidth]{images/weekend_vs_weekdays_routes.png}
    \caption{Porovnanie priemernej vyťaženosti jednotlivých cyklotrás v rámci dňa počas pracovných dní a víkendu.}
    \label{weekday_vs_weekend_routes}
\end{figure}


\subsection{Porovnanie počas zimných a letných dní}
Výrazné rozdiely možno pozorovať aj pri porovnaní zimnej a letnej sezóny. V zime je počet cyklistov výrazne nižší, čo súvisí s nepriaznivými poveternostnými podmienkami, nižšími teplotami a často aj s horším stavom ciest a chodníkov. Dáta ukazujú, že v zime sú 2 vrcholy vyťaženosti počas pracovného týždňa bližšie pri sebe, čo bude pravdepodobne dôsledkom kratších dní. Počas výkendov, počet cyklistov narastá s hodinou dňa a po trinástej hodine začína klesať, pričom po dvadsiatej hodine na cyklotrasách nestretneme takmer žiadneho cyklistu.


\begin{figure}[H]
    \centering
    \includegraphics[width=0.7\linewidth]{images/summer_winter.png}
    \caption{Porovnanie priemernej vyťaženosti jednotlivých cyklotrás v rámci dňa počas pracovných dní a víkendu.}
    \label{winter_vs_summer}
\end{figure}

%summer/winter ratio

\begin{figure}[H]
    \centering
    \includegraphics[width=0.9\linewidth]{images/summer_winter_routes.png}
    \caption{Porovnanie priemernej vyťaženosti jednotlivých cyklotrás v rámci dňa počas pracovných dní a víkendu.}
    \label{winter_vs_summer_routes}
\end{figure}

\section{Vyťaženosť počas jednotlivých mesiacov}

\begin{figure}[H]
    \centering
    \includegraphics[width=0.7\linewidth]{images/seasonality_by_month.png}
    \caption{Porovnanie priemernej vyťaženosti jednotlivých cyklotrás v rámci dňa počas pracovných dní a víkendu.}
    \label{seasinality_by_month}
\end{figure}


\section{Prediktívny model}





