\documentclass{report}
\usepackage[utf8]{inputenc}
\usepackage{graphicx}
\usepackage[utf8]{inputenc}
\usepackage[IL2]{fontenc}
\usepackage[slovak]{babel}
\usepackage[margin=3cm,
            top=1.5cm,
            includeheadfoot,
            footskip=1cm]{geometry}
\usepackage{amssymb}
\usepackage{amsmath}
\usepackage{hyperref}
\usepackage{pbox}
\usepackage{float}
\usepackage{setspace}
\usepackage{array}
\newcommand{\code}[1]{\texttt{#1}}
\newcommand{\bigone}{\mbox{\normalfont\Large\bfseries 1}}
\renewcommand{\thechapter}{\arabic{chapter}}

\newcommand*{\vertbar}{\rule[-1ex]{0.5pt}{2.5ex}}
\newcommand*{\horzbar}{\rule[.5ex]{2.5ex}{0.5pt}}
\newenvironment{lbmatrix}[1]
  {\left[\array{{c}*{#1}{c}@{}}}
  {\endarray\right]}
\begin{document}
  \begin{titlepage}
    \begin{center}
        \vspace*{1cm}
            
        \Huge
        \textbf{Analýza dát zo cyklosčítačov v Bratislave}
            
        \vspace{0.5cm}
        \LARGE
        Tím: Jednotné programovacie družstvo a.s.

\vspace{0.5cm}
\LARGE 

Lucia Ganajová

Tuan Dávid Nguyen Van

Jakub Novotný

Marek Šugár
        

        \vfill
        
        \Large

        
        Fakulta matematiky, fyziky a informatiky\\
        Univerzita Komenského v Bratislave\\
        2026
            
    \end{center}
\end{titlepage}

\chapter*{Úvod}

V posledných rokoch je z hľadiska urbánneho plánovania badať tendencie
znižovania akejsi nadvlády áut v uliciach. Práve naopak, najmä v krajských mestách
je zvyšovaný dôraz na budovania infraštruktúr pre chodcov a cyklistov. Výnimkou nie je ani Bratislava,
v ktorej v posledných rokoch je možné pozorovať nárast práve prepravy po dvoch kolesách.
\newline
\par
Magistrát mesta aj z dôvodu ďalšej analýzy inštaloval od roku $2014$ na niektorých úsekoch cyklotrás 
sčítače, ktoré s hodinovou frekvenciou akumulujú absolútne početnosti cyklistov, ktorí v danom časovom intervale
prešli ktorýmkoľvek z dvoch smerov. Toto veľké množstvo dát vytvára priestor na hlbšie analýzy, ktoré bližšie
dokážu opísať správanie obyvateľov Bratislavy.
\newline
\par
V tomto reporte je cieľom poukázať na zaujímavé interpretačné dôsledky, vychádzajúce z dát. Dôraz je kladený 
interpretáciu dát s rôznou frekvenciou a spojitosti, resp. rozdielnosti medzi jednotlivými cyklotrasami. Okrem iného, hlbší
dôraz je kladený spojitosť celkovej vyťaženosti cyklotrás s počasím. 
\newline
\par
Report vznikol ako jeden z podkladov riešenia semestrálneho projektu v rámci predmetu \textbf{Princípy dátovej vedy}
na v zimnom semestri 2025-2026 na Fakulte matematiky, fyziky a informatiky Univerzity Komenského v Bratislave.


\begin{flushright}
\textit{v Bratislave, január 2026}
\par
autori
\end{flushright}

%% Toto sa bude dokončovať vzhľadom na obsah, ktorý bude

\tableofcontents

\clearpage

\chapter{Spojitosť vyťaženia cyklotrás a počasia}

Pre účely bližšieho porozumenia vplyvu stavu počasia v Bratislave na celkový 
počet absolútnych prejazdov po jednotlivých cyklotrasách je možné zvoliť viacero 
možných prístupov. V projekte bol zvolený prístup tréningu vybraných algoritmov strojového učenia
a následnej evaluácie predikovania modelov.

\section{Predspracovanie dát}

Dostupné dátové zdroje poskytujú s hodinovou frekvenciou informácie o počte prejazdov
v oboch smeroch na danej cyklotrase. Dátové zdroje akumulujúce údaje o stave počasie boli avšak 
dostupné v spoľahlivej podobe iba s dennou frekvenciou.
\newline
\par
Z tohoto dôvodu bolo vhodné agregovať dáta z jednotlivých dostupných hodín prejazdov do konkrétnych dní. Pre účely
priamočiarejšej a úspornejšej interpretácie má zmysel agregovať aj dáta z oboch smerov cyklotrasy dokopy – tým získavame
počet denných prejazdov na cyklotrase celkovo. Pre účely interpretácie vplyvu počasia je to z apriórneho hľadiska postačujúce,
z hľadiska dostupných dátových zdrojov jedna z mála možností.
\newline
\par
Počasie dátový zdroj zrkadlí pomocou viacerých čiastkových premenných, niektoré viacej aj menej interpretačne uchopiteľné. 
Z relatívne bohatého dátového zdroja je možné vybrať reprezentatívnu podvzorku premenných – využívame prístup interpretačnej
spojitosti na základe istého poznania správania.
\newline
\par
Do našej využitej množiny premenných zaraďujeme údaje o teplote – \textbf{priemerná teplota}, \textbf{minimálna} a \textbf{maximálna teplota}.
Na základe týchto dát je možné očakávať priamu úmeru s absolútnym počtom prejazdov cyklotrasou. Vyššie teploty počas dňa pravdepodobne indukujú
vyšší počet prejazdov, vyššiu tendenciu ľudí uprednostniť tento dopravný prostriedok pred napr. \textit{autom, MHD} a pod.
\newline
\par
Významný vplyv na vyťaženosť cyklotrasy a na s ním prepojený sentiment obyvateľov voči cyklodoprave má určite \textbf{spád zrážok}. Do databázy premenných boli 
vložené obe dostupné metriky spádu zrážok – spád dažďových a snehových zrážok.
\newline
\par
Okrem iného významnými metrikami sú \textbf{rýchlosť vetra} a \textbf{tlak}. Prvá spomenutá metrika má dozaista vplyv tendenciu ľudí
uprednostniť dopravu bicyklom, spoločne s tlakom vzduchu, ktorý má významný vplyv na vyvíjanú biozáťaž a tendenciu ľudí vykonávať fyzickú 
záťaž na rámec, takpovediac, \textit{nutnosti}.
\newline
\par
Pri delení dát do zodpovedajúcich tréningových, validačných a testovacích množín dát je nutné zachovať chronologickú následnosť dát,
keďže v podstate sa jedná o časové rady a neuváženým prístupom k deleniu je možné disponovať skreslednými, zväčša nadhodnotenými, výsledkami
výkonnosti modelov.
\newline
\par
Z dôvodu variabilnej dĺžky monitorovaného časového intervalu pre rôzne cyklotrasy nie je možné pristupovať k deľbe pomocou fixných časových breakpointov.
V projekte bol využitý prístup rozdelenia celkového intervalu v pomere $6:3:1$ pre tréningovú, validačnú a testovaciu množinu. Postupnosť dátových vstupov je zabezpečená
zachovaním vzájomnej následnosti jednotlivých podintervalov. Tréningová množina predstavuje prvých $60 \%$ dát, validačná nasledujúcich $30 \%$ a zvyšok testovacia množina. 

\section{Lineárna regresia}

Prvotnou ideou spomedzi klasických prístupov strojového učenia je lineárna regresia. Na tomto mieste je vhodné ozrejmiť aj metriku úspešnosti modelovania, 
na ktorú budeme výkonnosti modelov reflektovať. V kapitole je využitá \textbf{priemerná absolútna odchýlka}, normalizovaná zodpovedajúcim priemerom ako $\frac{MAE(Y, \hat{Y})}{\overline{Y}}$, 
kde $Y$ označuje skutočnú hodnotu prejazdov, $\hat{Y}$ predikcia modelu a $\overline{Y}$ priemer skutočných hodnôt. Normalizácia je
vhodná najmä z dôvodu relativizácie presnosti predikcie medzi rôznymi cyklotrasami. Metrika \textbf{priemernej absolútnej percentuálnej odchýlky (MAPE)} sa preukázala byť zväčša 
numericky nestabilná, pri veľmi nízkych skutočných počtoch prejazdov a i napriek tomu relatívne presnej predikcii metriky indikovala abnormálne vysoké hodnoty. 
\newline
\par
Výkonnosti modelovania pre jednotlivé cyklotrasy sú vizualizované na \textit{Obrázku 1}.

\begin{figure}[H]
    \centering
    \includegraphics[width=0.7\linewidth]{images/LR_mae.png}
    \caption{Presnosť modelovania počtu prejazdov cyklotrasou lineárnou regresiou.}
\end{figure}

Je možné pozorovať pomerne široké spektrum presností - nižšie aj vyššie odchýlky.

\section{Penalizované lineárne regresie}

\end{document}
